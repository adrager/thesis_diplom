The Standard Model (SM) of particle physics is one of the most successful scientific theories. It has been extraordinary successful, not only in describing observed interactions and the production of particles, but also predicting new particles.\\
Build on gauge groups, the SM has shown its predictive power to a high precision, especially in electroweak processes, which have been tested at LEP to a very high accuracy.
The prediction and later discovery of the $W^{\pm},Z$ boson, which mediate the weak force, and of the $b$ and $t$ quarks has further increased the credency in the Standard Model. Despite all successes, experimental and theoretical arguments suggest that the Standard Model can only be consider a low energy approximation of a more fundamental theory.\\
For example Supersymmetry is able to solve many problems of the Standard Model in an elegant and convenient way. One prediction of Supersymmetric models is the existence of new massive particles. Interpretation of experimental results and phenomenological arguments suggest particle masses around the TeV scale, placing them in reach of the Large Hadron Colliders (LHC).\\
One of the main goals of the LHC and its two main experiments ATLAS and CMS is the search for these non SM particles. This thesis describes a search for such particles in events with jets and missing transverse momentum at CMS. Evidently it is crucial to predict the rate of such events due to Standard Model processes such as the production and decay of \W bosons.
The main focus of this thesis therefore is on one of the background estimation methods used to predict SM events involving not identified electrons and muons originating from \W decays leading to such a signature. The sum of the SM background estimations contributing to this analysis, is compared to observed events in the luminosity of \lumi collected by CMS in 2011. No excess over the Standard Model expectation has been found. The obtained exclusion limits are among the most sensitive limits on SUSY up to date.\\
This thesis starts with a short introduction to the Standard Model and Supersymmetry (Sec.~\ref{sec:theory}) followed by a description of the LHC and the CMS detector and its components (Sec.~\ref{sec:detector}). Next the concept of jets with missing transverse momentum analysis is introduced (Sec.~\ref{sec:search_beyondSM}) followed by a detailed discussion of the lost-lepton method (Sec.~\ref{sec:lostlepton}). The last part describes the other contributing background estimations, closing with the interpretation of the results within the cMSSM and Simplified Models (Sec.~\ref{sec:ra2}).
