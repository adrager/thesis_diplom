In this thesis a method has been presented which is able to predict events including high energetic jets and undetected leptons from a $W$-decay in the final state. 
A muon control sample is selected on data and reweighted to predict the undetected leptons using reconstruction and isolation efficiencies obtained from MC and with a Tag \& Probe method on data. 
The method has been improved in the context of higher pileup, more statistics and with increased \HT and \MHT cuts. Tests have been performed to prove that the method is capable of estimating the background for very high pileup distributions and very high \HT and \MHT selections. These studies show that the method can be applied for the even higher pileup and \HT and \MHT cuts as expected for the 2012 data.\\ 
The method has been used together with the other background estimations in a search for events hinting to physics beyond the SM on the \lumi of data collected by CMS in 2011. Since no excess over SM expectations has been observed, limits have been calculated in the cMSSM and Simplified models excluding first and second generation squark and qluino masses up to $\tilde q \approx 1.2$GeV and $\tilde g \approx 650$GeV. These limits are among the world wide most sensitive limits published up to date.\\
The increased center of mass energy in 2012 of $\sqrt{s}=8$ TeV and the expected 14 TeV in 2014 will improve the sensitivity of this analyses even further in the higher $m_0, m_{1/2}$ plane of the cMSSM and also for the Simplified Models.\\
%To improve the results even further different parametrization of the efficiency for example in pileup activity can be studied.\\
Another focus has recently become increasingly important. The exclusion of a very large fraction of the $m_0 , m_{1/2}$ plane for first and second generation squarks suggests to focus on the search for third generation squarks. Third generation squark decays include often a b-jet motivating the implementation of a b-tag in order to select two independent samples one with a b-tag one without which would increase the sensitivity of this analysis.\\

 Despite the fact that a large space in the cMSSM model framework has been excluded the search for physics beyond the SM will continues and especially for the more challenging search for third generation squarks the hunt is far from over.